\documentclass[10pt]{article}
\usepackage[left=0.4in,right=0.4in,top=0.7in,bottom=0.4in]{geometry}
\usepackage{hyperref}
\usepackage{fancyhdr}
\usepackage{listings}
\usepackage{xcolor}
\usepackage{tocloft}
\usepackage{pdflscape}
\usepackage{multicol}
\usepackage{graphicx}
\usepackage[utf8]{inputenc}
\usepackage[T1]{fontenc}
\usepackage[vietnamese]{babel}

\renewcommand*{\ttdefault}{pcr}
\renewcommand\cftsecfont{\fontsize{8}{9}\bfseries}
\renewcommand\cftsecpagefont{\fontsize{8}{9}\mdseries}
\renewcommand\cftsubsecfont{\fontsize{5}{6}\mdseries}
\renewcommand\cftsubsecpagefont{\fontsize{5}{6}\mdseries}
\renewcommand\cftsecafterpnum{\vspace{-1ex}}
\renewcommand\cftsubsecafterpnum{\vspace{-1ex}}

\lstdefinestyle{shared}{
    belowcaptionskip=1\baselineskip,
    breaklines=true,
    xleftmargin=\parindent,
    showstringspaces=false,
    basicstyle=\fontsize{5.5}{6}\ttfamily,
}
\lstdefinestyle{cpp}{
	style=shared,
    language=C++,
    keywordstyle=\bfseries\color{green!40!black},
    commentstyle=\itshape\color{red!80!black},
    identifierstyle=\color{blue},
    stringstyle=\color{purple!40!black},
}
\lstdefinestyle{java}{
    style=shared,
    language=Java,
    keywordstyle=\bfseries\color{green!40!black},
    commentstyle=\itshape\color{purple!40!black},
    identifierstyle=\color{blue},
    stringstyle=\color{orange},
}
\lstdefinestyle{py}{
    style=shared,
    language=Python,
    keywordstyle=\bfseries\color{green!40!black},
    commentstyle=\itshape\color{purple!40!black},
    identifierstyle=\color{blue},
    stringstyle=\color{orange},
}
\lstdefinestyle{txt}{
    style=shared,
}
\lstset{escapechar=@}

\pagestyle{fancy}
\fancyhead[L]{Academy of Cryptography Techniques}
\fancyhead[R]{\thepage}
\fancyfoot[C]{}

\fancypagestyle{plain}
{
\fancyhead[L]{Academy of Cryptography Techniques}
\fancyhead[R]{\thepage}
\fancyfoot[C]{}
}

\title{\vspace{-4ex}\Large{Academy of Cryptography Techniques ACM-ICPC Notebook 2025}}
\author{}
\date{}

\begin{document}
\begin{landscape}
\begin{multicols}{2}

\maketitle
\vspace{-13ex}
\tableofcontents
\pagestyle{fancy}

\section{Initial Setup}
\subsection{Template}
\lstinputlisting[style=cpp]{c:/Users/HP PC/ICPC/algo_course/res/notebook/1.initial/initial.h}

\section{Graph}
\subsection{DFS}
\lstinputlisting[style=cpp]{c:/Users/HP PC/ICPC/algo_course/res/notebook/2.graph/1.dfs.txt}

\subsection{DAG (Directed Acyclic Graph)}
\lstinputlisting[style=cpp]{c:/Users/HP PC/ICPC/algo_course/res/notebook/2.graph/2.dag.txt}

\subsection{Euler Path}
\lstinputlisting[style=cpp]{c:/Users/HP PC/ICPC/algo_course/res/notebook/2.graph/3.euler.txt}

\subsection{Topological Sort}
\lstinputlisting[style=cpp]{c:/Users/HP PC/ICPC/algo_course/res/notebook/2.graph/4.topo.txt}

\subsection{Joints and Bridges}
\lstinputlisting[style=cpp]{c:/Users/HP PC/ICPC/algo_course/res/notebook/2.graph/5.jointAndBridges.txt}

\subsection{Strongly Connected Components (SCC)}
\lstinputlisting[style=cpp]{c:/Users/HP PC/ICPC/algo_course/res/notebook/2.graph/6.SCC.txt}

\subsection{BFS}
\lstinputlisting[style=cpp]{c:/Users/HP PC/ICPC/algo_course/res/notebook/2.graph/7.BFS.txt}

\subsection{Dijkstra}
\lstinputlisting[style=cpp]{c:/Users/HP PC/ICPC/algo_course/res/notebook/2.graph/8.dijkstra.txt}

\subsection{Disjoint Set Union and Kruskal}
\lstinputlisting[style=cpp]{c:/Users/HP PC/ICPC/algo_course/res/notebook/2.graph/9.disjointSetUnionsKruskal.txt}

\section{Math}
\subsection{Modular Arithmetic}
\lstinputlisting[style=cpp]{c:/Users/HP PC/ICPC/algo_course/res/notebook/3.math/modul.txt}

\subsection{Combinatorics}
\lstinputlisting[style=cpp]{c:/Users/HP PC/ICPC/algo_course/res/notebook/3.math/combinatoric.txt}

\section{Advanced Data Structures}
\subsection{Segment Tree}
\lstinputlisting[style=cpp]{c:/Users/HP PC/ICPC/algo_course/res/notebook/4.AdvancedDataStructures/1_Segment_Tree.txt}

\subsection{Lazy Propagation}
\lstinputlisting[style=cpp]{c:/Users/HP PC/ICPC/algo_course/res/notebook/4.AdvancedDataStructures/2_Lazy_Update.txt}

\subsection{Persistent Segment Tree}
\lstinputlisting[style=cpp]{c:/Users/HP PC/ICPC/algo_course/res/notebook/4.AdvancedDataStructures/3.Persistent_Seg_Tree.cpp}

\subsection{Fenwick Tree}
\lstinputlisting[style=cpp]{c:/Users/HP PC/ICPC/algo_course/res/notebook/4.AdvancedDataStructures/4_Fenwick_Tree.cpp}

\subsection{Trie}
\lstinputlisting[style=cpp]{c:/Users/HP PC/ICPC/algo_course/res/notebook/4.AdvancedDataStructures/5_Trie.cpp}

\end{multicols}
\end{landscape}

% Cheat sheets (bỏ comment nếu có file pdf)
% \centerline{\includegraphics[trim={0 0 0 3cm}, clip, width=\textwidth]{cheatsheet/p01.pdf}}
% \centerline{\includegraphics[trim={0 0 0 3cm}, clip, width=\textwidth]{cheatsheet/p02.pdf}}
% ... thêm các trang khác tương tự ...

\end{document}
\end{document}