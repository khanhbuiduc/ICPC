\documentclass[10pt]{article}
\usepackage[left=0.4in,right=0.4in,top=0.7in,bottom=0.4in]{geometry}
\usepackage{hyperref}
\usepackage{fancyhdr}
\usepackage{listings}
\usepackage{xcolor}
\usepackage{tocloft}
\usepackage{pdflscape}
\usepackage{multicol}
\usepackage{graphicx}

\renewcommand*{\ttdefault}{pcr}
\renewcommand\cftsecfont{\fontsize{8}{9}\bfseries}
\renewcommand\cftsecpagefont{\fontsize{8}{9}\mdseries}
\renewcommand\cftsubsecfont{\fontsize{5}{6}\mdseries}
\renewcommand\cftsubsecpagefont{\fontsize{5}{6}\mdseries}
\renewcommand\cftsecafterpnum{\vspace{-1ex}}
\renewcommand\cftsubsecafterpnum{\vspace{-1ex}}

\lstdefinestyle{shared}{
    belowcaptionskip=1\baselineskip,
    breaklines=true,
    xleftmargin=\parindent,
    showstringspaces=false,
    basicstyle=\fontsize{5.5}{6}\ttfamily,
}
\lstdefinestyle{cpp}{
	style=shared,
    language=C++,
    keywordstyle=\bfseries\color{green!40!black},
    commentstyle=\itshape\color{red!80!black},
    identifierstyle=\color{blue},
    stringstyle=\color{purple!40!black},
}
\lstdefinestyle{java}{
    style=shared,
    language=Java,
    keywordstyle=\bfseries\color{green!40!black},
    commentstyle=\itshape\color{purple!40!black},
    identifierstyle=\color{blue},
    stringstyle=\color{orange},
}
\lstdefinestyle{py}{
    style=shared,
    language=Python,
    keywordstyle=\bfseries\color{green!40!black},
    commentstyle=\itshape\color{purple!40!black},
    identifierstyle=\color{blue},
    stringstyle=\color{orange},
}
\lstdefinestyle{txt}{
    style=shared,
}
\lstset{escapechar=@}

\pagestyle{fancy}
\fancyhead[L]{Academy of Cryptography Techniques}
\fancyhead[R]{\thepage}
\fancyfoot[C]{}

\fancypagestyle{plain}
{
\fancyhead[L]{Academy of Cryptography Techniques}
\fancyhead[R]{\thepage}
\fancyfoot[C]{}
}

\title{\vspace{-4ex}\Large{Academy of Cryptography Techniques ACM-ICPC Notebook 2025}}
\author{}
\date{}

\begin{document}
\begin{landscape}
\begin{multicols}{2}

\maketitle
\vspace{-13ex}
\tableofcontents
\pagestyle{fancy}

\section{Initial Setup}
\subsection{Template}
\lstinputlisting[style=cpp]{c:/Users/HP PC/ICPC/algo_course/res/notebook/1.initial/initial.h}

\section{Graph}
\subsection{DFS}
\begin{lstlisting}[style=cpp]
void dfs(int u)
{
    visited[u] = true;
    for (auto v: adj[u])
    {
        if(!visited[v])
            dfs(v);
    }
}
\end{lstlisting}

\subsection{DAG (Directed Acyclic Graph)}
\begin{lstlisting}[style=cpp]
int trace[100005], state[100005];
bool haveCircle = false;

void dfs_dag(int u)
{
    state[u] = 1;
    for (int v : adj[u])
    {
        if (state[v] == 0)
        {
            trace[v] = u;
            dfs_dag(v);
        }
        else if (state[v] == 1)
        {
            print(u, v);
            haveCircle = true;
            exit(0);
        }
    }
    state[u] = 2;
}

void print(int start, int end)
{
    vector<int> ans;
    ans.push_back(end);
    for (int cur = start; cur != end; cur = trace[cur])
        ans.push_back(cur);
    ans.push_back(end);
    cout << ans.size() << endl;
    reverse(ans.begin(), ans.end());
    for (auto i : ans)
        cout << i << ' ';
}
\end{lstlisting}

\subsection{Euler Path}
\begin{lstlisting}[style=cpp]
// All vertices in graph have even degree for Euler cycle
set<int> adj[100005];
vector<int> ans;

void dfs_euler(int u)
{
    for(auto v : adj[u])
    {
        adj[v].erase(u);
        adj[u].erase(v);
        dfs_euler(v);
    }
    ans.push_back(u);
}
\end{lstlisting}

\subsection{Dijkstra}
\begin{lstlisting}[style=cpp]
vector<int> dijkstra(vector<vector<pair<int,int>>>& adj, int start) {
    int n = adj.size();
    vector<int> dist(n, INF);
    dist[start] = 0;
    priority_queue<pair<int,int>, 
                  vector<pair<int,int>>,
                  greater<pair<int,int>>> pq;
    pq.push({0, start});
    
    while (!pq.empty()) {
        auto [d, u] = pq.top(); pq.pop();
        if (d > dist[u]) continue;
        
        for (auto [v, w] : adj[u]) {
            if (dist[u] + w < dist[v]) {
                dist[v] = dist[u] + w;
                pq.push({dist[v], v});
            }
        }
    }
    return dist;
}
\end{lstlisting}

\section{Math}
\subsection{Number Theory}
\begin{lstlisting}[style=cpp]
// GCD and LCM
int gcd(int a, int b) {
    return b ? gcd(b, a % b) : a;
}

int lcm(int a, int b) {
    return a / gcd(a, b) * b;
}

// Modular inverse
int mod_inv(int a, int m) {
    int g = gcd(a, m);
    if (g != 1) return -1;
    return mod_pow(a, m - 2, m);
}

// Fast power
int mod_pow(int base, int exp, int mod) {
    int result = 1;
    base %= mod;
    while (exp > 0) {
        if (exp & 1) result = (1LL * result * base) % mod;
        base = (1LL * base * base) % mod;
        exp >>= 1;
    }
    return result;
}
\end{lstlisting}

\subsection{Convex Hull}
\begin{lstlisting}[style=cpp]
// Graham scan algorithm
struct Point {
    int x, y;
    bool operator<(Point p) const {
        return x < p.x || (x == p.x && y < p.y);
    }
};

// Cross product
int cross(const Point& O, const Point& A, const Point& B) {
    return (A.x - O.x) * (B.y - O.y) - (A.y - O.y) * (B.x - O.x);
}

vector<Point> convex_hull(vector<Point>& points) {
    int n = points.size(), k = 0;
    if (n <= 3) return points;
    vector<Point> hull(2*n);
    
    sort(points.begin(), points.end());
    
    // Lower hull
    for (int i = 0; i < n; i++) {
        while (k >= 2 && cross(hull[k-2], hull[k-1], points[i]) <= 0) k--;
        hull[k++] = points[i];
    }
    
    // Upper hull
    int t = k + 1;
    for (int i = n - 2; i >= 0; i--) {
        while (k >= t && cross(hull[k-2], hull[k-1], points[i]) <= 0) k--;
        hull[k++] = points[i];
    }
    
    hull.resize(k-1); // Last point is same as first
    return hull;
}
\end{lstlisting}

\section{Advanced Data Structures}
\subsection{Disjoint Set Union (DSU)}
\begin{lstlisting}[style=cpp]
struct DSU {
    vector<int> p, sz;
    DSU(int n): p(n), sz(n,1) {
        iota(p.begin(), p.end(), 0);
    }
    int find(int x) {
        return (p[x]==x ? x : p[x]=find(p[x]));
    }
    bool unite(int a,int b){
        a=find(a); b=find(b);
        if(a==b) return false;
        if(sz[a]<sz[b]) swap(a,b);
        p[b]=a; sz[a]+=sz[b];
        return true;
    }
};
\end{lstlisting}

\subsection{Segment Tree}
\begin{lstlisting}[style=cpp]
struct SegTree {
    vector<int> tree;
    int n;
    
    SegTree(int n) : n(n) {
        tree.resize(4*n);
    }
    
    void update(int node, int start, int end, 
                int idx, int val) {
        if (start == end) {
            tree[node] = val;
        } else {
            int mid = (start + end) / 2;
            if (idx <= mid)
                update(2*node, start, mid, idx, val);
            else
                update(2*node+1, mid+1, end, idx, val);
            tree[node] = tree[2*node] + tree[2*node+1];
        }
    }
    
    int query(int node, int start, int end, 
              int l, int r) {
        if (r < start || end < l) return 0;
        if (l <= start && end <= r) return tree[node];
        int mid = (start + end) / 2;
        return query(2*node, start, mid, l, r) + 
               query(2*node+1, mid+1, end, l, r);
    }
};
\end{lstlisting}

\end{multicols}
\end{landscape}

% Cheat sheets (bỏ comment nếu có file pdf)
% \centerline{\includegraphics[trim={0 0 0 3cm}, clip, width=\textwidth]{cheatsheet/p01.pdf}}
% \centerline{\includegraphics[trim={0 0 0 3cm}, clip, width=\textwidth]{cheatsheet/p02.pdf}}
% ... thêm các trang khác tương tự ...

\end{document}
\end{document}
